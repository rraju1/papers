\documentclass{article}
\usepackage{amsmath}
\title{Houdini: Fooling Deep Structured Prediction Models}
\date{12-28-2018}
\author{Ravi Raju}

\begin{document}
\maketitle
\pagenumbering{gobble}
% \newpage
\pagenumbering{arabic}
\section{Motivation}
This paper introduces a framework called Houdini for producing adverserial examples not exclusive to classification. Some of the problems that are posed in the introduction is that some of these attacks are based on gradients, which may be a combinatorial problem to attack. These are focusing on structured prediction tasks.
\section{Solution}
In the abstract attacks with Houdini achieve higher success rate than those based on the traditional surrogates used to train these models.
\section{Results}

\section{Discussion/Takeaway}


\end{document}

% \begin{equation*}
% 	f(x) = x^2
% \end{equation*}